\documentclass{report}

\usepackage{amsmath}
\usepackage{bytefield}
\usepackage{hyperref}

\begin{document}

\title{Apple Array System ABI}
\author{V. E. McHale}
\maketitle

\tableofcontents

\section{Arrays}

Apple arrays are completely flat, consisting of three fields (rank, dimensions, data) laid out contiguously. Rank and dimension are 64-bit integers; data are stored in row-major order.

\subsection{Elementwise Access}

For an array of rank $r$ with dimensions $d_1,\ldots,d_r$ the element $x_{i_1i_2\cdots i_r}$ is at the offset $\displaystyle \sum_{k=1}^r s_k i_k$ where $\displaystyle s_k=\prod_{n=k+1}^r d_n$

\subsection{Example}

A {\tt 2x4x3} Apple array is laid out in memory like so:

\begin{bytefield}{10}
    \\
    \bitheader{0-27} \\
    \bitbox{1}{3} & \bitbox{3}{2 4 3} & \bitbox{3}{$a_{00i}$} & \bitbox{3}{$a_{01i}$} & \bitbox{3}{$a_{02i}$} & \bitbox{3}{$a_{03i}$} & \bitbox{3}{$a_{10i}$} & \bitbox{3}{$a_{11i}$} & \bitbox{3}{$a_{12i}$} & \bitbox{3}{$a_{13i}$}
\end{bytefield}

\section{Tuples}

Tuples are stack-allocated when it is not possible to pass them in registers. A tuple with an array as one of its elements is not flat but rather contains a pointer to the array.

\end{document}
